\documentclass{article}
\usepackage[a4paper, total={6.9in, 8in}]{geometry}
\usepackage[utf8]{inputenc}
\usepackage{noweb}
\usepackage{titlesec}
\usepackage{hyperref}
\hypersetup{
    colorlinks=true,
    linkcolor=blue,
    filecolor=magenta,
    urlcolor=blue,
}
\usepackage{fontspec}
\usepackage{librefranklin}
\setsansfont{Linux Libertine}
\setmonofont[Scale=0.9]{lmmono12-regular.otf}


%\renewcommand\familydefault{\sfdefault}
%
\titleformat*{\section}{\Large\bfseries\sffamily}
\titleformat*{\subsection}{\large\bfseries\sffamily}
\titleformat*{\subsubsection}{\itshape\bfseries\sffamily}
\begin{document}

We organise the R files as follows: {\tt{}main.R} is the main entry point of everything: it sources other files
and actually load the data, call the functions, and produces graphs and confidence intervals numbers and so on.
Other files are no-side-effect definitions of functions, which should be called in {\tt{}main.R} -- except {\tt{}readdat.R},
which actually read and preprocesses the \textsc{csv} files and store the clean data as data frames in the current R
environment.

\section{Libraries}

The following libraries are needed to run this analysis:

\nwfilename{main.Rnw}\nwbegincode{1}\moddef{main.R}\endmoddef
suppressPackageStartupMessages(\{
  library('stringr')
  library('purrr')
  library('reshape2')
  library('gtools')
  library('lattice')
  library('tibble')
  library('latticeExtra')
  library('RColorBrewer')
  library('outliers')
\})
\nwendcode{}\nwbegindocs{2}\nwdocspar

\section{Configuring file paths}
All file paths to the data files goes to {\tt{}filepath-def.R}. The file should be self-explanatory.

\nwenddocs{}\nwbegincode{3}\moddef{filepath-def.R}\endmoddef
deathfile='../dat/eurostat-20210527T1802.csv'
popufile='../dat/2020-06-22_population.csv'
##covidswefile='../dat/covid-sweden-2020-09-14.csv'
##polishfile='../dat/polish-20200924T1733.csv'
##swedcovidregionfile='../dat/Folkhalsomyndigheten_covid19_modified_2021-02-07.rds'
## Härje Widing's COVID-19 death data
coviddeathfiles=list(
  'BE'='../dat/Härje-covid-snapshot-20210531T1502/COVID-19-BELGIUM-level-2.csv',
  'DE'='../dat/Härje-covid-snapshot-20210531T1502/COVID-19-GERMANY-level-2.csv',
  'ES'='../dat/Härje-covid-snapshot-20210531T1502/COVID-19-SPAIN-level-2.csv',
  'CH'='../dat/Härje-covid-snapshot-20210531T1502/COVID-19-SWITZERLAND-level-2.csv',
  'UK'='../dat/Härje-covid-snapshot-20210531T1502/COVID-19-UK-level-2.csv'
)
\nwendcode{}\nwbegindocs{4}\nwdocspar

\section{Loading the data into the R environment}

To load and clean all the data, we source the file {\tt{}readdat2.R}. This file should produces important data frames
in the current R environment such as {\tt{}mf}, {\tt{}mf.1}, {\tt{}mf.2} etc., which contains the excess death per age-group per
region. the trailing {\tt{}.1} and {\tt{}.2} represents NUTS level. For example, {\tt{}mf} contains country level such as {\tt{}SE};
and {\tt{}mf.1} contains {\tt{}SE1}, {\tt{}SE2} and so on.

\nwenddocs{}\nwbegincode{5}\moddef{main.R}\plusendmoddef
source('readdat2.R')
\nwendcode{}\nwbegindocs{6}\nwdocspar

In the remaining of this section, we implement this {\tt{}readdat2.R}.

\subsection{Convert CSV files into suitable R data frame}

The {\tt{}readdat2.R} is layed out as follows:

\nwenddocs{}\nwbegincode{7}\moddef{readdat2.R}\endmoddef
\LA{}Initialise and include libraries\RA{}
\LA{}Read EuroStat death and population data\RA{}
\LA{}Merge the EuroStat death and population data for convenience\RA{}
\LA{}Read COVID-19 death data by Härje Widing\RA{}
\LA{}Define country-age-group mapping for COVID-19 data\RA{}
\nwendcode{}\nwbegindocs{8}\nwdocspar

First, we include the libraries and initialisation. Note that we set the
{\tt{}stringAsFactors} option to true to change the behaviour of R's readcsv
function. This is labelled deprecated in R, but in 2021 it is still usable.

\nwenddocs{}\nwbegincode{9}\moddef{Initialise and include libraries}\endmoddef
suppressPackageStartupMessages(\{
  library('stringr')
  library('purrr')
  library('reshape2')
  library('tibble')
\})

source('utils.R')        # Various utility functions
source('filepath-def.R')
cat('Preprocessing data...\\n')
\nwendcode{}\nwbegindocs{10}\nwdocspar

The EuroStat death and population data is stored in two big, seperate, \textsc{CSV} files
\nwenddocs{}\nwbegincode{11}\moddef{Read EuroStat death and population data}\endmoddef
deathraw = read.csv(deathfile)
popuraw = read.csv(popufile)
# Typo in the CSV file, I guess
deathraw$age[deathraw$age=='NK']   = 'UNK'
deathraw$age[deathraw$age=='OTAL'] = 'TOTAL'
# I want level names in death matches populatuion
for (i in seq_along(deathraw$age))
    deathraw$age[i] = str_replace(deathraw$age[i], '-', '_')

\nwendcode{}\nwbegindocs{12}\nwdocspar

We merge the two data sets so that the death and population of the same age-region-week
(abbrev. \textsc{arw}) is on the same row. This is convenient because most of the time
we select our data by \textsc{arw}.
\nwenddocs{}\nwbegincode{13}\moddef{Merge the EuroStat death and population data for convenience}\endmoddef
loclvl = function (l) nchar(l) - 2                               # NUTS code to NUTS level
yr     = function (s) str_extract(s, '\\\\d\{4\}')                   # Get year from XyyyyWww format
wkno   = function (s) str_remove(str_extract(s, 'W\\\\d\{2\}'), 'W') # Get week from XyyyyWww format

mkmf = function (deathraw, popu, lvl) \{
  allloc = unique(deathraw$geo.time)     # All available NUTS codes
  loc = allloc[loclvl(allloc) %in% lvl]  # Target NUTS codes
  agegrp = unique(deathraw$age)          # Target age group
  sex    = c('F','M')                            # Target sex. we don't use 'T'
  availmask     = deathraw$age %in% agegrp &
                  deathraw$geo.time %in% loc &
                  deathraw$sex %in% sex
  deathraw      = data.frame(deathraw)           # Copy the data frame before changing it.
  deathraw      = deathraw[availmask,]
  deathraw.mlt    = melt(deathraw, id.vars=c('sex', 'age', 'geo.time'))
  deathraw.mlt$yr   = yr(deathraw.mlt$variable)
  deathraw.mlt$popyr= ifelse(deathraw.mlt$yr == '2020', '2019', deathraw.mlt$yr)
  deathraw.mlt$wkno = wkno(deathraw.mlt$variable)
  popu.mlt = melt(popu, id.vars=c('sex', 'age', 'geo.time'))
  popu.mlt$variable = yr(popu.mlt$variable)
  colnames(popu.mlt)[colnames(popu.mlt) == 'variable'] = 'popyr'

  # Inner-joining the two tables
  tmp.mlt = merge(popu.mlt, deathraw.mlt, by = c('sex','age','geo.time','popyr'),
                  suffixes=c('.popu','.death'))
  mdf = tmp.mlt[tmp.mlt$sex == 'M',];  fdf = tmp.mlt[tmp.mlt$sex == 'F',];
  mdf$sex      = NULL;                  fdf$sex      = NULL;
  mdf$variable = NULL;                  fdf$variable = NULL;
  mdf$popyr    = NULL;                  fdf$popyr    = NULL;
  rm(tmp.mlt); rm(popu.mlt); rm(deathraw.mlt)
  # Inner-join again
  mf.mlt = merge(mdf, fdf,
                 by = c('age','geo.time','yr', 'wkno'),
                 suffixes = c('.m', '.f'))
  # Replace the "90" age group to "90_Inf", for later processing
  mf.mlt[['age']][which(mf.mlt[['age']] == '90')] = '90_Inf'
  mf.mlt
\}
mf = mkmf(deathraw, popuraw, 0)
mf.1 = mkmf(deathraw, popuraw, 1)
mf.2 = mkmf(deathraw, popuraw, 2)
mf.3 = mkmf(deathraw, popuraw, 3)
\nwendcode{}\nwbegindocs{14}\nwdocspar

On the other hand, Widing's data set is a list of \textsc{csv} files. The ``level 2''
CSV contains national-level per age-sex-day \textsc{covid}-19 mortality. Here we convert it
to per-week and also clean up the age group for our purpose.

\nwenddocs{}\nwbegincode{15}\moddef{Read COVID-19 death data by Härje Widing}\endmoddef
coviddeathraw = lapply(coviddeathfiles, read.csv2)
\LA{}Exclude Switzerland and Germany\RA{}
normalise_härje_data = function (coviddeathraw) \{
    lapply(coviddeathraw, function (dframe) \{
        dframe = \LA{}Normalise age groups in national COVID data\RA{}
        aggregate(dframe[c('male_covid_death','female_covid_death','total_covid_death')],
                  by  = dframe[c('WeekNo','AgeGrp_min','AgeGrp_max','Year')],
                  FUN = sum)
    \})
\}
coviddeath = normalise_härje_data(coviddeathraw)
\nwendcode{}\nwbegindocs{16}\nwdocspar

In the next section, we will implement the age group normalisation algorithm, as well
as other age-group-related processing. But first, notice that there is an error in the Switzerland
level 2 data set, in which for the age group ``80+'' there are two different representations:
\begin{center}
\begin{tabular}{ c c }
 \texttt AgeGrp\_min & \texttt AgeGrp\_max \\
 "80" & NA  \\
 "80+" & "80+"
\end{tabular}
\end{center}
Also, all other countries has integer age groups but Switzerland has strings. So we simply exclude
Switzerland at this moment. Germany, on the other hand, has a problem that, in the EuroStat data
set there is no per-age population size; therefore we don't have enough information to interpret
the \textsc{covid} mortality count.
\nwenddocs{}\nwbegincode{17}\moddef{Exclude Switzerland and Germany}\endmoddef
coviddeathraw[['CH']] = NULL
coviddeathraw[['DE']] = NULL
\nwendcode{}\nwbegindocs{18}\nwdocspar

\section{Processing age groups from different data sets}

The EuroStat data and \textsc{covid} death data set has different definitions of age groups.
EuroStat has a fine-grained age group, each of which has spans five years, from 0\~4,
5\~9, 10\~14, and so on, until 90+. On the other hand, the \textsc{covid}-19 data set have different
age group for different countries, but fortunately, all of their beginning ages and
end ages's second digits are 0, 4, and 9 respectively. In other words, the EuroStat
age groups are ``sub-partition'' of that of the \textsc{covid}-19 death data for all countries.

Therefore, for each country, we can simply leave the \textsc{covid}-19 data set's age group 'untouched',
but normalised in format, and then later on merge the corresponding EuroStat sub groups' death
counts. This results in different age groups in different countries -- we may normalise it further
later on when we do statistical testing or visualisation--but at least for visualisation, this
normalisation gives us the more fine-grained age group that we can get out of the data.

But first, we define an utility function which check if a row exists exactly
a data frame.

\nwenddocs{}\nwbegincode{19}\moddef{utils.R}\endmoddef
rowexists = function (row, df) nrow(merge(row, df))>0
\nwendcode{}\nwbegindocs{20}\nwdocspar

\nwenddocs{}\nwbegincode{21}\moddef{Normalise age groups in national COVID data}\endmoddef
\{
    ## Produce an data frame of unique age group with two columns:
    ## 'AgeGrp_min' and 'AgeGrp_max'
    ## Remove the row if both min and max are NA
    dframe = data.frame(dframe)
    dframe = dframe[!(is.na(dframe[['AgeGrp_min']]) &
                      is.na(dframe[['AgeGrp_max']])),]
    ## In UK there is an 0~1 and 1~4 group. Merge them in this case.
    dframe.ageonly = dframe[c('AgeGrp_min','AgeGrp_max')]
    if (rowexists(list('AgeGrp_min'=0,'AgeGrp_max'=1), dframe.ageonly) &&
        rowexists(list('AgeGrp_min'=1,'AgeGrp_max'=4), dframe.ageonly))\{
        which_zeroone = which(dframe[['AgeGrp_min']] == 0 &
                              dframe[['AgeGrp_max']] == 1)
        which_onefour = which(dframe[['AgeGrp_min']] == 1 &
                              dframe[['AgeGrp_max']] == 4)
        dframe[which_zeroone,'AgeGrp_max']      = 4
        dframe[which_onefour,'AgeGrp_min'] = 0   # We'll sum them up later
    \}
    ## If AgeGrp_max is NA and min is not then replace NA with infinity
    which_veryold = which((!is.na(dframe[['AgeGrp_min']])) &
                          is.na(dframe[['AgeGrp_max']]))
    dframe[['AgeGrp_max']][which_veryold] = Inf
    ## If AgeGrp_min is NA and max is not then replace NA with 0
    which_baby = which((!is.na(dframe[['AgeGrp_max']])) &
                        is.na(dframe[['AgeGrp_min']]))
    dframe[['AgeGrp_min']][which_baby] = 0
    dframe
\}
\nwendcode{}\nwbegindocs{22}\nwdocspar

It is also convenient to have a list which contains all the age groups of various countries,
so the testing or visualisation routine can find use different age group depending on the
country.

\nwenddocs{}\nwbegincode{23}\moddef{utils.R}\plusendmoddef
unique_row = function (df) df[!duplicated(df),]
\nwendcode{}\nwbegindocs{24}\nwdocspar

\nwenddocs{}\nwbegincode{25}\moddef{Define country-age-group mapping for COVID-19 data}\endmoddef
country_age_group_map = lapply(coviddeath, function (dframe) \{
    X = unique_row(dframe[c('AgeGrp_min','AgeGrp_max')])
    rownames(X) = NULL
    X
\})
\nwendcode{}\nwbegindocs{26}\nwdocspar

Because the EuroStat data has different age group, sooner or later we will need to
combine the sub-partition of EuroStat according to the country level age group. The
following is a high-level function which, given a \textsc{covid}-19 age group (one of the elements
in the variable {\tt{}country{\char95}age{\char95}group{\char95}map}), returns the corresponding EuroStat age
groups.

\nwenddocs{}\nwbegincode{27}\moddef{utils.R}\plusendmoddef
agegrp_tostr = function (minag,maxag)
    paste(minag, maxag, sep='_')

make_eurostat_age_group_map = function (coarse_map) \{
    minag = (0:18)*5
    maxag = minag + 4
    maxag[length(maxag)] = Inf
    usedmask = integer(length(minag))
    result = list()
    for (j in seq_len(nrow(coarse_map))) \{
        included = list()
        for (i in seq_along(minag)) \{
            if (coarse_map[j,'AgeGrp_min'] <= minag[i] && coarse_map[j,'AgeGrp_max'] >= maxag[i]) \{
                included[[length(included)+1]] = agegrp_tostr(minag[i], maxag[i])
                usedmask[i] = 1
            \}
        \}
        result[[j]] = unlist(included)
    \}
    if (prod(usedmask) != 1)
        warning('make_eurostat_age_group_map: Supplied age group does not span all possible ages')
    names(result) = mapply(agegrp_tostr, coarse_map[['AgeGrp_min']], coarse_map[['AgeGrp_max']])
    result
\}
\nwendcode{}\nwbegindocs{28}\nwdocspar


\section{Iterators to Loop through Pairs of Binomial Data}

The tests we conduct have a general structure.  Fixing an \textsc{arw} combination, let $Y_{i}$, $Y_{C}$, $Y^*$ be,
respectively, the all-cause male death count of Year $i$ in which \textsc{covid}-19 were absent, male death count officially
announced as due to \textsc{covid}-19, and the amount by which the male death count of 2020 exceeds that of a baseline. Also let $n_{i}$,
$n_C$, and $n^*$ be the corresponding total, sex-aggregated death count (male plus female); with, optionally, $(m_M,m_F)$
and $(r_M,r_F)$, which is the male and female population size of the excess death's population and the \textsc{covid}-19
death's population.

The file {\tt{}arw-iterators.R} defines high-order functions which calls
\begin{equation}
f(Y_1,\cdots,Y_n,n_1,\cdots,n_n,Y_{C},n_C,Y^*,,n^*,r_M,r_F,m_M,m_F)
\end{equation}
for some $f$ and returns a list of lists, each of which contains the \textsc{arw} and the function's evaluation result.

\nwenddocs{}\nwbegincode{29}\moddef{arw-iterators.R}\endmoddef
iter_ageweek = function (target_geo, get_m, get_r,
                         get_history_death, get_covid_death, get_excess_death,
                         fn,
                         age_groups, ...) \{
  excess_nonpos   = list() # These 3 var. contains all error ARW.
  excess_notavail = list()
  covid_notavail  = list()
  wkno_uniq = as.character(1:52)
  results   = list()
  for (age in age_groups) \{
    for (wk in wkno_uniq) \{
      covidgrp_popsiz.m    =get_r(target_geo, age, wk, 'M')
      covidgrp_popsiz.f    =get_r(target_geo, age, wk, 'F')
      excessgrp_popsiz.m   =get_m(target_geo, age, wk, 'M')
      excessgrp_popsiz.f   =get_m(target_geo, age, wk, 'F')
      ## The following are vectors of numbers, each element represent the
      ## death count of a year. Year is encoded as names of the vectors
      death.m.history      =get_history_death(target_geo, age, wk, 'M')
      death.f.history      =get_history_death(target_geo, age, wk, 'F')
      death.m.excess       =get_excess_death(target_geo, age, wk, 'M')
      death.f.excess       =get_excess_death(target_geo, age, wk, 'F')
      death.m.covid        =get_covid_death(target_geo, age, wk, 'M')
      death.f.covid        =get_covid_death(target_geo, age, wk, 'F')
      \LA{}Check death toll consistency\RA{}
      obj = fn(death.m.history,   death.m.history + death.f.history,
               death.m.covid,     death.m.covid + death.f.covid,
               death.excess,      death.excess,
               covidgrp_popsiz.m, covidgrp_popsiz.f,
               excessgrp_popsiz.m, excessgrp_popsiz.f,
               ...)
      info = list(age = age, geo = target_geo, wk = wk, result=obj)
      results[[length(results)+1]] = info
    \}
  \}
  class(results) = c('iter_ageweek_result')
  attr(results,'miss') = list(excess_nonpos=excess_nonpos,
                              excess_notavail=excess_notavail,
                              covid_notavail=covid_notavail)
  results
\}
\nwendcode{}\nwbegindocs{30}\nwdocspar

Sometimes excess death can be negative, depending on what baseline one chooses; or
it can be \textsc{na} if the data set itself contains \textsc{na} for whatever reasons. When this
happens, we want to record them and continue to next iteration.

\nwenddocs{}\nwbegincode{31}\moddef{Check death toll consistency}\endmoddef
      if (is.na(death.m.excess) || is.na(death.f.excess)) \{
        excess_notavail[[length(excess_notavail)+1]] = c(wk = wk, age = age)
        cat(sprintf('Excess death not available, skip (wkno, age) = (%s,%s) \\n', wk, age))
        next
      \} else if (!(death.m.excess > 0 && death.f.excess > 0)) \{
        excess_nonpos[[length(excess_nonpos)+1]] = c(wk = wk, age = age)
        cat(sprintf('Non-positive excess, skip (wkno,age,excess.m,excess.f) = (%s,%s,%f,%f)\\n',
                    wk, age, death.m.excess, death.f.excess))
        next
      \}
      if (is.na(death.m.covid) || is.na(death.f.covid)) \{
        covid_notavail[[length(covid_notavail)+1]] = c(wk = wk, age = age)
        cat(sprintf('COVID-19 death not available, skipping (wkno, age) = (%s,%s) \\n', wk, age))
        next
      \}
\nwendcode{}\nwbegindocs{32}\nwdocspar

\subsection{Extract death counts from the pre-processed Eurostat data frame}

In order to use the {\tt{}iter{\char95}ageweek} function defined above, it is neccessary to
supply it with the {\tt{}get{\char95}*} functions.
\nwenddocs{}\nwbegincode{33}\moddef{arw-iterators.R}\plusendmoddef
\LA{}Define get historical death function for EuroStat\RA{}
\LA{}Define get excess death function for EuroStat\RA{}
\LA{}Define get m function for EuroStat\RA{}
\nwendcode{}\nwbegindocs{34}\nwdocspar
where ``get m'' $m$ refers to the population size of the excess population, as discussed above.

The Eurostat data frames {\tt{}mf.*} is
sufficient to provide the functions {\tt{}get{\char95}history{\char95}death} and {\tt{}get{\char95}excess{\char95}death}.


Now let's define a function which get returns the historical death count. The names
{\tt{}eshist} stands for Eurostat history.
\nwenddocs{}\nwbegincode{35}\moddef{Define get historical death function for EuroStat}\endmoddef
mk_eshist_getter = function (mf, agegrp_map, fromwhichyear = 2020) \{
    function (target_geo, age, wk, sex) \{
        idx = which(mf$age %in% agegrp_map[[age]] &
                    mf$geo.time == target_geo &
                    mf$wkno == sprintf('%02d', as.integer(wk)) &
                    as.integer(mf$yr) < fromwhichyear)
        if (length(idx) == 0) \{ return(NA); \}
        key = if (sex == 'M') 'value.death.m' else 'value.death.f'
        mfaggr = aggregate(mf[[key]][idx], list(mf[['yr']][idx]),
                           sum, SIMPLIFY=T)
        yrorder = order(mfaggr[['Group.1']])
        result = mfaggr[['x']][yrorder]
        yrname = paste0('Y', mfaggr[['Group.1']][yrorder])
        names(result) = yrname
        result
    \}
\}
\nwendcode{}\nwbegindocs{36}\nwdocspar

Note that the {\tt{}mf.*} data frames from {\tt{}readdat.R} instead of {\tt{}mfexc.*} should
be passed to the above function. And the excess death data is similar. Notice that
the result is rounded to the nearest integer.

\nwenddocs{}\nwbegincode{37}\moddef{Define get excess death function for EuroStat}\endmoddef
mk_esexcess_getter = function (mf, agegrp_map, whichyear=2020, baseline_fn=mean) \{
    get_history = mk_eshist_getter(mf)
    function (target_geo, age, wk, sex) \{
        baseline = baseline_fn(get_history(target_geo, age, wk, sex))
        if (is.na(baseline)) return(NA)
        idx = which(mf$age %in% agegrp_map[[age]] &
                    mf$geo.time == target_geo &
                    mf$wkno == sprintf('%02d', as.integer(wk)) &
                    as.integer(mf$yr) == whichyear)
        if (length(idx) == 0) \{ return(NA); \}
        key = if (sex == 'M') 'value.death.m' else 'value.death.f'
        death_total = sum(mf[[key]][idx])
        round(death_total - baseline)
    \}
\}
\nwendcode{}\nwbegindocs{38}\nwdocspar

The Eurostat database contains the population size of both male and female
at all NUTS regions. This information is stored in {\tt{}popuraw} data frame
produced by {\tt{}readdat.R}. We can use this to implement the {\tt{}get{\char95}m} argument
of {\tt{}iter{\char95}ageweek}.

\nwenddocs{}\nwbegincode{39}\moddef{Define get m function for EuroStat}\endmoddef
mk_m_getter = function (mf, agegrp_map, whichyear=2020) \{
    function (target_geo, age, wk, sex) \{
        idx = which(mf$age %in% agegrp_map[[age]] &
                    mf$geo.time == target_geo &
                    mf$wkno == sprintf('%02d', as.integer(wk)) &
                    as.integer(mf$yr) == whichyear)
        if (length(idx) == 0) \{ return(NA); \}
        key = if (sex == 'M') 'value.popu.m' else 'value.popu.f'
        total = sum(mf[[key]][idx])
        total
    \}
\}
\nwendcode{}\nwbegindocs{40}\nwdocspar

\subsection{Extract death counts from the pre-processed COVID-19 data frame}

Because of the different data sources, different countries might need a different processing functions. But
in \href{https://github.com/harjew/COVID-19-Data/tree/master/covid-data}{this GitHub repository} by Härje Widing, the
format of these data are normalised from all the data sources except for the \textsc{NUTS} code.

\subsubsection{Sweden}
Due to the lack of clean regional data, we will use the national \textsc{covid}-19 mortality first. Here we implement
the {\tt{}get{\char95}covid{\char95}death} and {\tt{}get{\char95}r} funcions needed by the iterator using the national-level data.

\nwenddocs{}\nwbegincode{41}\moddef{arw-iterators.R}\plusendmoddef
\LA{}Define get covid death function\RA{}
\LA{}Define get r function\RA{}
\nwendcode{}\nwbegindocs{42}\nwdocspar

First, we define a utility function which converts any regional-level NUTS
codes to country code. Say, ``SE2'' should map to ``SE'', and so on.

\nwenddocs{}\nwbegincode{43}\moddef{utils.R}\plusendmoddef
nationalise_NUTS = function (code)
    substr(code, start = 1, stop = 2)
\nwendcode{}\nwbegindocs{44}\nwdocspar

Also it is convenient to have a function which splits age group strings like
``90\_Inf'' to {\tt{}c(90,Inf)}.

\nwenddocs{}\nwbegincode{45}\moddef{utils.R}\plusendmoddef
agegrp_tonum = function (agegrp_str) \{
    ans = sapply(strsplit(agegrp_str, '_')[[1]], as.numeric, simplify=T)
    names(ans) = c('AgeGrp_min','AgeGrp_max')
    ans
\}
\nwendcode{}\nwbegindocs{46}\nwdocspar

In the following functions, all NUTS codes are converted to national-level
in this manner. These data-getters does not need to be supplied with
age group, as it expects the argument {\tt{}age} matches the actual age group.

\nwenddocs{}\nwbegincode{47}\moddef{Define get covid death function}\endmoddef
mk_national_covid_death_getter = function (coviddeath, whichyear=2020) \{
    function (target_geo, age, wk, sex) \{
        national_geo = nationalise_NUTS(target_geo)
        if (is.null(coviddeath[[national_geo]])) \{
            stop(sprintf('We do not have national-level COVID mortality of "%s"',
                          national_geo))
        \}
        agegrp_num = agegrp_tonum(age)
        whichrow = which(coviddeath[[national_geo]][['AgeGrp_min']] == agegrp_num[['AgeGrp_min']] &
                         coviddeath[[national_geo]][['AgeGrp_max']] == agegrp_num[['AgeGrp_max']] &
                         coviddeath[[national_geo]][['WeekNo']] == wk &
                         coviddeath[[national_geo]][['Year']] == whichyear)
        if (length(whichrow) == 0) \{
            stop(sprintf(
                  'covid_death_getter: no record for (Geo, Age_min, Age_max, Week, Year) = (%s,%s,%s,%s,%s)',
                  target_geo, agegrp_num[['AgeGrp_min']], agegrp_num[['AgeGrp_max']], wk, whichyear))
        \} else if (length(whichrow) > 1) \{
            stop(sprintf(
                  'covid_death_getter: >1 records for (Geo, Age_min, Age_max, Week, Year) = (%s,%s,%s,%s,%s)',
                  target_geo, agegrp_num[['AgeGrp_min']], agegrp_num[['AgeGrp_max']], wk, whichyear))
        \}
        key = if (sex == 'M') 'male_covid_death' else 'female_covid_death'
        coviddeath[[national_geo]][[key]][whichrow]
    \}
\}
\nwendcode{}\nwbegindocs{48}\nwdocspar

Similar to the ``get\_m'' function, we define the ``get\_r'' function which returns the population
size. But because this function uses the EuroStat population size, there is a need to pass it
the age group map. The function takes, as arguments, the \textsc{covid}-19 age group, maps it
to EuroStat age group, and return the total of population size. Any non-country-level NUTS
code passed to the function will be converted to country-level; in other words, the returned
number will always be the national-level population size.

\nwenddocs{}\nwbegincode{49}\moddef{Define get r function}\endmoddef
mk_national_r_getter = function (mf, agegrp_map, whichyear=2020) \{
    function (target_geo, age, wk, sex) \{
        national_geo = nationalise_NUTS(target_geo)
        idx = which(mf$age %in% agegrp_map[[age]] &
                    mf$geo.time == national_geo &
                    mf$wkno == sprintf('%02d', as.integer(wk)) &
                    as.integer(mf$yr) == whichyear)
        if (length(idx) == 0) \{ return(NA); \}
        key = if (sex == 'M') 'value.popu.m' else 'value.popu.f'
        total = sum(mf[[key]][idx])
        total
    \}
\}
\nwendcode{}\nwbegindocs{50}\nwdocspar

\subsection{Testing out the iterator}

The following is a test to see if the iterator actually works.

\nwenddocs{}\nwbegincode{51}\moddef{TEST-iter.R}\endmoddef
source('arw-iterators.R')
iter_ageweek('SE1', get_m, get_r,
             get_history_death, get_covid_death, get_excess_death,
             fn,
             age_group_map = EUROSTAT_AGEGROUP_MAP, ...)
\nwendcode{}\nwbegindocs{52}\nwdocspar



\section{Bayesian test of sex ratio}

\nwenddocs{}\nwbegincode{53}\moddef{bayesratio.R}\endmoddef
## set.seed(777)
set.seed(5201314)

source('filepath-def.R')
source('sedat.R')
source('mixture-ci.R')

suppressPackageStartupMessages(\{
    library('lattice')
    library('latticeExtra')
\})
## TODO: write this!
\nwendcode{}\nwbegindocs{54}\nwdocspar

\section{Confidence interval of the mixture model}

The mixture code is in the file {\tt{}mixture-ci.R}.
\nwenddocs{}\nwbegincode{55}\moddef{mixture-ci.R}\endmoddef
suppressPackageStartupMessages(\{
  library(lattice)
  library(latticeExtra)
  library(RColorBrewer)
  library(parallel)
\})
Lstat = Vectorize(function (p,alpha,nc=100,ns=130,nsamp=400000) \{
  yc = rbinom(n=nsamp, size=nc, p=p)
  ys = rbinom(n=nsamp, size=ns, p=p)
  Tall = (yc+ys)/(nc+ns)
  logTall = log(yc+ys) - log(nc+ns)
  Tc = yc/nc
  logTc = log(yc)-log(nc)
  Ts = ys/ns
  logTs = log(ys)-log(ns)
  logL = yc * (logTc - logTall) + ys*(logTs - logTall) +
    (nc-yc)*(log(1-Tc) - log(1-Tall)) +
    (ns-ys)*(log(1-Ts) - log(1-Tall))
  k = quantile(logL, probs=alpha,na.rm=T)
  names(k) = NULL
  k
\}, 'p')

# xs = seq(0.08, 0.92, length.out=200)
# Q = Lstat(xs,0.95)
# crit = max(Q)
# plot(y=Q, x=xs, type='l', xlab = expression(p_C), ylab='95th quantile of lik. ratio stat.')

pwrfn = (function (pn, pc, alpha, c, nc=100, ns=130, nsamp=100000) \{
  yc = rbinom(n=nsamp, size=nc, p=pc)
  ys = rbinom(n=nsamp, size=ns, p=alpha*pn + (1-alpha)*pc)
  Tall = (yc+ys)/(nc+ns)
  logTall = log(yc+ys) - log(nc+ns)
  Tc = yc/nc
  logTc = log(yc)-log(nc)
  Ts = ys/ns
  logTs = log(ys)-log(ns)
  logL = yc * (logTc - logTall) + ys*(logTs - logTall) +
        (nc-yc)*(log(1-Tc) - log(1-Tall)) +
        (ns-ys)*(log(1-Ts) - log(1-Tall))
  sum(logL>c)/nsamp
\})
cmapply <- function(FUN, ..., MoreArgs = NULL, SIMPLIFY = TRUE,
                    USE.NAMES = TRUE)
\{
  l <- expand.grid(..., stringsAsFactors=FALSE)
  r <- do.call(mapply, c(
    list(FUN=FUN, MoreArgs = MoreArgs, SIMPLIFY = SIMPLIFY, USE.NAMES = USE.NAMES),
    l
  ))
  if (is.matrix(r)) r <- t(r)
  cbind(l, r)
\}

##pwr = cmapply(pn=seq(0.15, 0.85, by=0.05), pc=seq(0.15,0.85,by=0.04), alpha=seq(-0.5,1.5,by=0.1),
##         FUN= function (pn, pc, alpha) \{
##            print(c(pn,pc,alpha))
##            pwrfn(pn, pc, alpha, c=1.98)
##          \})
##colnames(pwr) = c('pN', 'pC', 'alpha', 'pwr')

# my.settings <- canonical.theme(color=FALSE)
# my.settings[['strip.background']]$col <- 'white'
# my.settings[['strip.border']]$col<- 'black'
# levelplot(pwr~pC*alpha|pN, data=pwr, panel = panel.2dsmoother, col.regions = colorRampPalette(brewer.pal(11,'Spectral'),2),
#           par.settings = my.settings,
#           strip = strip.custom(strip.levels = c(TRUE, TRUE)),
#           par.strip.text=list(col='black'))
#
#
#
# xs2 = seq(0.08, 0.92, length.out=200)
# Q2 = Lstat(xs2,0.95,nc=400,ns=600,nsamp=400000)
# crit2 = max(Q2)
# plot(y=Q2, x=xs2, type='l', xlab = expression(p_C), ylab='95th quantile of lik. ratio stat.')
# pwr2 = cmapply(pn=seq(0.15, 0.85, by=0.05), pc=seq(0.15,0.85,by=0.04), alpha=seq(-0.5,1.5,by=0.1),
#               FUN= function (pn, pc, alpha) \{
#                 print(c(pn,pc,alpha))
#                 pwrfn(pn, pc, alpha, c=1.96, nc=400, ns=600)
#               \})
# colnames(pwr2) = c('pN', 'pC', 'alpha', 'pwr')
# levelplot(pwr~pC*alpha|pN, data=pwr2, panel = panel.2dsmoother, col.regions = colorRampPalette(brewer.pal(11,'Spectral'),2),
#           par.settings = my.settings,
#           strip = strip.custom(strip.levels = c(TRUE, TRUE)),
#           par.strip.text=list(col='black'))

loglik_mainpart = function (pn,pc,a,yn,yc,ys,nn,nc,ns) \{
  yn*log(pn) + (nn-yn)*log(1-pn) + yc*log(pc) + (nc-yc)*log(1-pc) + ys*log(a*pn+(1-a)*pc) +
      (ns-ys)*log(1-a*pn-(1-a)*pc)
\}
mle_alpharestricted = function (a0,yn,yc,ys,nn,nc,ns) \{
  obj = function (p) - loglik_mainpart(p[1],p[2],a0,yn,yc,ys,nn,nc,ns)
  res = simpleError('Encountered infinite value?')
  counter <- 1
  max_tries <- 1000
  while(inherits(res, 'error') & counter < max_tries) \{
    res <- tryCatch(\{ optim(runif(2,min=0.001,max=0.999),
                            obj,
                            method='L-BFGS-B',
                            lower=c(0.001,0.001),upper=c(0.999,0.999)) \},
                    error = function(e) e)
    counter <- counter + 1
  \}
  if (inherits(res, 'error'))
    stop('Cannot optimise alpha-restricted likelihood')
  else
    res
\}

logL_alpharestricted = function (a0,yn,yc,ys,nn,nc,ns) \{
  mle_H0_optobj = tryCatch(\{
    mle_alpharestricted(a0,yn,yc,ys,nn,nc,ns)
  \}, error = function (cond) \{ NULL \})
  if (is.null(mle_H0_optobj)) return(NA)
  if (mle_H0_optobj[['convergence']] != 0) return(NA)
  mle    = loglik_mainpart(yn/nn, yc/nc, (nn*(ns*yc - nc*ys))/(ns*(nn*yc - nc*yn)), yn,yc,ys,nn,nc,ns)
  mle + mle_H0_optobj$value
\}

critval_logL_alpharestricted = function (a0,nn,nc,ns, nsamp=3000) \{
  ## Estimate of distribution of logL_alpharestricted under different values of the null hypothesis.
  ## For all possible pN and pC, simulate logL_alpharestricted and take quantiles.
  cmapply(pn=seq(0.08, 0.92, by=0.05), pc=seq(0.08, 0.92,by=0.04),
          FUN= function (pn, pc) \{
            print(c(pn,pc))
            yn = rbinom(n=nsamp, size=nn, p=pn)
            yc = rbinom(n=nsamp, size=nc, p=pc)
            ys = rbinom(n=nsamp, size=ns, p=a0*pn+(1-a0)*pc)
            S = mapply(function (yn,yc,ys) \{
              tryCatch(\{
                logL_alpharestricted(a0,yn,yc,ys,nn,nc,ns)
              \}, error=function (cond) NA)
            \},yn,yc,ys)
            quantile(S, prob = 0.95, na.rm=T)
          \})
\}

# nntest = 600;  yntest = 300
# nctest = 600;  yctest = 550
# nstest = 200;  ystest = 170

#levelplot(r~pn*pc, data=ct)


## Given an observed Y and n, compute an approximate
## confidence interval from the data.
alpha_ci = function (lvl, yn,yc,ys,nn,nc,ns, fineness=1000) \{
  a0_candidates = seq(-0.3, 1.5, length.out = fineness)
  critval = qchisq(lvl, df=1)/2
  rejected = integer(fineness)
  for (i in seq_along(a0_candidates)) \{
    LL = logL_alpharestricted(a0_candidates[i], yn,yc,ys,nn,nc,ns)
    rejected[i] = as.integer(LL > critval)
  \}
  whichzero = which(rejected==0)
  lowerbidx = whichzero[1]
  upperbidx = whichzero[length(whichzero)]
  c(lower = a0_candidates[lowerbidx],
    upper = a0_candidates[upperbidx])
\}
# rej = alpha_ci(0.95, yntest,yctest,ystest,nntest,nctest,nstest)

## Find coverage probability
covg_prob = function (lvl,pn,pc,a,nn,nc,ns, nsamp=600) \{
  yn = rbinom(n=nsamp, size=nn, p=pn)
  yc = rbinom(n=nsamp, size=nc, p=pc)
  ys = rbinom(n=nsamp, size=ns, p=a*pn+(1-a)*pc)
  edgecases = (yn == 0) | (yn == nn) | (yc == 0) | (yc == nc) | (yn/nn == yc/nc)
  yn = yn[!edgecases]
  yc = yc[!edgecases]
  ys = ys[!edgecases]
  len_actual = sum(!edgecases)
  i = 0
  covg = mcmapply(function (yn,yc,ys) \{
    print(c(finished=i, total=nsamp))
    i <<- i+1
    ci = alpha_ci(lvl, yn,yc,ys,nn,nc,ns)
    (a > ci[1]) || (a < ci[2])
  \},yn,yc,ys, mc.cores=6)
  sum(covg,na.rm=T)/len_actual
\}
# covg_prob(0.95, 0.55,0.85,0.9,nntest,nctest,nstest)
# cprob_all = sapply(seq(0,1,length.out=20), function (alpha) \{
#   print(alpha)
#   covg_prob(0.95, 0.55,0.85,alpha,nntest,nctest,nstest)
# \})
\nwendcode{}\nwbegindocs{56}\nwdocspar
\end{document}
\nwenddocs{}
